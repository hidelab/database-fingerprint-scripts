\documentclass{article}
\usepackage{Sweave}
\usepackage{amsmath}
\usepackage{amscd}
\usepackage[tableposition=top]{caption}
\usepackage{ifthen}
\usepackage[utf8]{inputenc}
\usepackage{hyperref}
\usepackage[usenames]{color}
\definecolor{midnightblue}{rgb}{0.098,0.098,0.439}
\DefineVerbatimEnvironment{Sinput}{Verbatim}{xleftmargin=2em, fontshape=sl,formatcom=\color{midnightblue}}
\DefineVerbatimEnvironment{Soutput}{Verbatim}{xleftmargin=2em}
\DefineVerbatimEnvironment{Scode}{Verbatim}{xleftmargin=2em}
\fvset{listparameters={\setlength{\topsep}{0pt}}}
\renewenvironment{Schunk}{\vspace{\topsep}}{\vspace{\topsep}}

\setkeys{Gin}{width=\textwidth}
\begin{document}
\title{Brain subtypes}
\author{Gabriel Altschuler}
\maketitle
In this document we will test the ability of the fingerprint to pull out simliar arrays across brain-subtypes.
\\ The required data and metadata is contained within the R package \verb@pathprint@. We will use the \verb@pathprint.v.0.3.beta4@ build in this session. 
\\ First we need to source the pathprint package and load the data libraries containing the fingerprint and the metadata. We will also load the tissue-specific data from a local file from the shared directory on \verb@hpc111@.
\begin{Schunk}
\begin{Sinput}
> library(pathprint.v0.3.beta4)
> data(GEO.metadata.matrix)
> humanBrain<-readLines(
   "/Users/GabrielAltschuler/Documents/Projects/Fingerprinting/Brain/data/HumanBrian.txt"
   )
> humanBrain<-humanBrain[grep("GSM", humanBrain)]
> HumanBrainSamples<-GEO.metadata.matrix[GEO.metadata.matrix$GSM %in% humanBrain,]
> ratBrain<-readLines(
   "/Users/GabrielAltschuler/Documents/Projects/Fingerprinting/Brain/data/RatBrain.txt"
   )
> ratBrain<-ratBrain[grep("GSM", ratBrain)]
> RatBrainSamples<-GEO.metadata.matrix[GEO.metadata.matrix$GSM %in% ratBrain,]
> mouseBrain<-readLines(
   "/Users/GabrielAltschuler/Documents/Projects/Fingerprinting/Brain/data/mouseBrian.txt"
   )
> mouseBrain<-mouseBrain[grep("GSM", mouseBrain)]
> MouseBrainSamples<-GEO.metadata.matrix[GEO.metadata.matrix$GSM %in% mouseBrain,]
> brain.data<-rbind(MouseBrainSamples, HumanBrainSamples, RatBrainSamples)
\end{Sinput}
\end{Schunk}
We will define a set of Brain types that we are interesting in using as classifiers
\begin{Schunk}
\begin{Sinput}
> brainTypes<-c("Caudate Nucleus", "Cerebellum", "Frontal Cortex", "substantia nigra", "Hypothalamus", "pituitary", "amygdala", "hippocampus", "putamen", "pineal")
> brain.data$Type<-NA
> for (i in brainTypes){
   ID<-grep(i, apply(
               brain.data[, c("Title", "Source", "Characteristics")],
               1, function(y){
                         paste(y, collapse = " ")
                         }
                     ), ignore.case = TRUE)
   
   if(sum(!(is.na(brain.data$Type[ID]))) > 0) print(
     list(a = paste("double assignment for tissue", i, "at "),
          b = brain.data[ID, "GSM"][!(is.na(brain.data$Type[ID]))]))
   brain.data$Type[ID][!is.na(brain.data$Type[ID])] <- NA
   brain.data$Type[ID][is.na(brain.data$Type[ID])] <- i
   }
\end{Sinput}
\begin{Soutput}
$a
[1] "double assignment for tissue hippocampus at "

$b
[1] "GSM53302" "GSM53303" "GSM53304" "GSM53308" "GSM53309"
[6] "GSM53310"
\end{Soutput}
\begin{Sinput}
> brainTypes.table<-as.data.frame(table(brain.data$Type), stringsAsFactors = FALSE)
> colnames(brainTypes.table)<-c("Type", "n")
> brainTypes.table$nMouse<-sapply(brainTypes.table$Type,
       function(x){sum(grepl("Mus musculus", subset(brain.data,
                                               Type == x)$Species))})
> brainTypes.table$nRat<-sapply(brainTypes.table$Type,
       function(x){sum(grepl("Rattus norvegicus", subset(brain.data,
                                               Type == x)$Species))})
> brainTypes.table$nHuman<-sapply(brainTypes.table$Type,
       function(x){sum(grepl("Homo sapiens", subset(brain.data,
                                               Type == x)$Species))})
\end{Sinput}
\end{Schunk}
We will select sub-types that are represented by at least 2 arrays in each species. These will be used to construct a PCA to assess to what extent the different brain subtypes are distinguished relative to the species separation.
% latex table generated in R 2.11.1 by xtable 1.5-6 package
% Fri Jul 29 17:48:39 2011
\begin{table}[tbp]
\begin{center}
\begin{tabular}{rlrrrr}
  \hline
 & Type & n & nMouse & nRat & nHuman \\ 
  \hline
1 & Caudate Nucleus & 147 &   0 &   0 & 147 \\ 
  2 & Cerebellum & 214 &  50 &   2 & 162 \\ 
  3 & Frontal Cortex & 200 &  41 &   0 & 159 \\ 
  4 & Hypothalamus &  66 &  20 &  13 &  33 \\ 
  5 & amygdala & 168 &  20 & 135 &  13 \\ 
  6 & hippocampus & 387 &  98 & 274 &  15 \\ 
  7 & pineal &  26 &   0 &  26 &   0 \\ 
  8 & pituitary & 128 &   6 &  99 &  23 \\ 
  9 & putamen &   9 &   0 &   0 &   9 \\ 
  10 & substantia nigra &  25 &   0 &   0 &  25 \\ 
   \hline
\end{tabular}
\caption{Brain types and curated arrays}
\label{brainType}
\end{center}
\end{table}\end{document}
