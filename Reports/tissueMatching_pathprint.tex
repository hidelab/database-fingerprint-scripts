\documentclass{article}
\usepackage{Sweave}
\usepackage{amsmath}
\usepackage{amscd}
\usepackage[tableposition=top]{caption}
\usepackage{ifthen}
\usepackage[utf8]{inputenc}
\usepackage{hyperref}
\usepackage[usenames]{color}
\definecolor{midnightblue}{rgb}{0.098,0.098,0.439}
\DefineVerbatimEnvironment{Sinput}{Verbatim}{xleftmargin=2em, fontshape=sl,formatcom=\color{midnightblue}}
\DefineVerbatimEnvironment{Soutput}{Verbatim}{xleftmargin=2em}
\DefineVerbatimEnvironment{Scode}{Verbatim}{xleftmargin=2em}
\fvset{listparameters={\setlength{\topsep}{0pt}}}
\renewenvironment{Schunk}{\vspace{\topsep}}{\vspace{\topsep}}
\usepackage{lscape}

\begin{document}
\input{tissueMatching_pathprint-concordance}
\title{Matching arrays using the pathway fingerprint}
\author{Gabriel Altschuler}
\maketitle
In this document we will test the ability of the pathway fingerprint to pull out simliar arrays across platforms and species in the GEO corpus.
\\ The required data and metadata is contained within the R package \verb@pathprint@. In addition, we will make use of the tissue samples curated in \emph{Zilliox and Irizarry. A gene expression bar code for microarray data. Nat Meth (2007) vol. 4 (11) pp. 911-3}, a set of arrays from 6 tissues; muscle, lung, spleen, kidney, liver and brain.
\\ First we need to source the pathprint package and load the data libraries containing the fingerprint and the metadata. We will also load the tissue-specific data from a local file.
\begin{Schunk}
\begin{Sinput}
> library(pathprint)
> tissue.meta<-read.delim(
+   "/data/shared/Fingerprint/curatedCellTypes/barcode_figure2_data.txt",
+   stringsAsFactors = FALSE)