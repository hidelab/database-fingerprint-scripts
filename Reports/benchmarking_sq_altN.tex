\documentclass{article}
\usepackage{Sweave}
\usepackage{amsmath}
\usepackage{amscd}
\usepackage[tableposition=top]{caption}
\usepackage{ifthen}
\usepackage[utf8]{inputenc}
\usepackage{hyperref}
\usepackage[usenames]{color}
\definecolor{midnightblue}{rgb}{0.098,0.098,0.439}
\DefineVerbatimEnvironment{Sinput}{Verbatim}{xleftmargin=2em, fontshape=sl,formatcom=\color{midnightblue}}
\DefineVerbatimEnvironment{Soutput}{Verbatim}{xleftmargin=2em}
\DefineVerbatimEnvironment{Scode}{Verbatim}{xleftmargin=2em}
\fvset{listparameters={\setlength{\topsep}{0pt}}}
\renewenvironment{Schunk}{\vspace{\topsep}}{\vspace{\topsep}}

\begin{document}
\title{Benchmarking the pathway fingerprint - squared mean rank - hpc111 version}
\author{Gabriel Altschuler}
\maketitle

In this document the ability of the fingerprint to classify samples across species will be evaluated. Two datasets will be used, tissue samples, and cell types in the hematopoiesis lineage.
\section{Cross-species tissue identification}
A list of tissue-specific datasets was taken from \emph{Zilliox and Irizarry. A gene expression bar code for microarray data. Nat Meth (2007) vol. 4 (11) pp. 911-3}. This is composed of arrays from 6 tissues; muscle, lung, spleen, kidney, liver and brain.
\\ First we need to load the probability of expression (POE) matrix that corresponds to this dataset, referred to as the \emph{barcode dataset}, and the accompanying metadata. N.B. The POE matrix is the pathway data matrix prior to applying a threshold to produce the ternary fingerprint.
\begin{Schunk}
\begin{Sinput}
> library(pathprint)
> load(
+  "/home/galtschu2/Documents/Projects/Fingerprinting/data/sq_.POE.matrix.2011-07-27.RData"
+   )